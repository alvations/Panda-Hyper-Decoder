\chapter{Background}
\label{chap:background}

This chapter provides an literature review of the related work from the field of Machine Translation (Section 2.1 and 2.2), the use of lexical information in Statistical Machine Translation (Section 2.3), a survey of the state-of-art terminology extraction (Section 2.4.1 and 2.4.2) and ontology induction techniques (Section 2.4.3).

The \textit{most relevant background} to the work described in the following chapters are:

\begin{itemize}
\item \textbf{Section 2.2} on Phrase-based Machine Translation\footnote{This is the main machine translation paradigm we used in this thesis.}
\item \textbf{Section 2.3.7} on an overview of integrating additional lexical information in SMT
\item \textbf{Section 2.4.1} on a brief survey of term extraction techniques and 
\item \textbf{Section 2.4.2} that motivates the importance of terminology in machine translation
\item \textbf{Section 2.4.3} on a brief survey of ontology induction techniques
\item \textbf{Section 2.4.4} describing the phrased-based SMT configuration used throughout the thesis\footnote{Unless explicitly stated in the prose, all SMT experiments described in this thesis used this phrase-based configuration.}
\end{itemize}

\newpage
% Chapter 2.1
%%%%%%%%%%%%%%%%%%%%%%%%%%%%%%%%%%%%%%%%%%%%%%%%
% SMT
%%%%%%%%%%%%%%%%%%%%%%%%%%%%%%%%%%%%%%%%%%%%%%%%
\section{Statistical Machine Translation}

Machine Translation (MT) is the computational task to automatically translate between human languages.

The history of automatic translation traces back to  17th century ideas of a universal language at the demise  of Latin as the global \emph{lingua franca}
%The motivation was to create a \emph{``rational"} or \emph{``logic"} system of terse and unambiguous international communication that supersede natural language communication 
\citep{hutchins2000early}.

Before the invention of modern day computing machines, attempts were made to create \emph{mechanical dictionaries} that tried to map words/ideas/concepts into numerical code to mediate between languages. The most influential approach is perhaps Leibniz's (1714) monadic theory that encapsulates symbolic thoughts and formal logic \citep{busche2009gottfried} and assumes that cross-lingual understanding between languages is a matter of mapping natural language into monads. In one of his memos, \emph{The Art of Discovery} (1685), Leibniz wrote:

\blockquote{\emph{This [monadic] language will be the greatest instrument of reason [for] when there are disputes among persons, we can simply say: Let us calculate, without further ado, and see who is right.}}

In the early days of modern day machine translation in the 1960s, \citeauthor{becher1962mechanischen}'s \citeyearpar{becher1962mechanischen} work titled \emph{`Zur mechanischen Sprachübersetzung: ein Programmierungversuch aus dem Jahre 1661'} expresses that the historical ``mechanical dictionary" approaches foreshadowed certain principles of machine translation. The aim of \emph{mechanical dictionaries} are not unlike the modern day natural language processing (NLP) task of creating a multilingual ontology and word/concept mapping like Open Multilingual WordNet (OMW) \citep{bond2012survey} and multilingual Ontonotes \citep{weischedel2010ontonotes}.

In the 1930s, patents for a \emph{``general purpose translation using a mechanical multilingual dictionary"}  and \emph{``mechanical translations via universal grammatical functions"} were granted by France and Russia to Georges Artsrouni and Petr Trojanski respectively. The Statistical Machine Translation (SMT) paradigm is largely attributed to Warren Weavers memorandum to the Rockfella foundation in 1949 \citep{weaver1955translation}:

\blockquote{\emph{It is very tempting to say that a book written in Chinese is simply a book written in English which was coded into the ``Chinese code". If we have useful methods for solving almost any cryptographic problem, may it not be that with proper interpretation we already have useful methods for translation?}}

Data-driven SMT has developed rapidly with the introduction of IBM word alignment models (\citealp{brown1990statistical,brown1993mathematics}) and a \emph{word-based} SMT model which translates word-for-word. The word-based model was later superseded by  phrase-based models (\citealp{ochneymaxentsmt,marcu2002phrase,zens2002phrase,koehn2003statistical,koehn2004pharaoh}) which rely on the word-alignment methods developed by its word-based predecessor \citep{al1999statistical}.

%%%%%%%%%%%%%%%%%%%%%%%%%%%%%%%%%%%%%%%%%%%%%%%%
% PB-SMT
%%%%%%%%%%%%%%%%%%%%%%%%%%%%%%%%%%%%%%%%%%%%%%%%
\subsection{Phrase-Based SMT}

Phrase-Based Statistical Machine Translation (PB-SMT) models translate contiguous sequences of words from the source sentence to contiguous words in the target language. In this case, the term \emph{phrase} does not refer to the linguistic notion of syntactic constituent but the notion of \ngrams. \cite{knight1999decoding} showed that decoding word/phrase-based models involve search problems that grow exponentially with sentence length.

Phrase-based models significantly improve on word-based models, and work especially well for closely-related languages. This is mainly due to the modelling of local reordering and the assumption that most orderings of contiguous \ngrams are monotonic. However, this is not the case for translation between language pairs with divergent syntactic constructions; e.g. when translating between SVO-SOV languages. 

\cite{tillmann2004unigram} and \cite{al2006distortion} proposed several sophisticated lexicalized reordering and distortion models to address most long-distance reordering issues. Alternatively, to overcome reordering issues with a simple distortion penalty, \cite{zollmann2008systematic} memorized a larger phrase \ngram sequence from very large training data and allow larger distortion limits; it achieves similar results to more sophisticated reordering techniques with fewer training data. In practice, reordering is set to a small window and \cite{birch2010metrics} showed that phrase-based models tend to perform poorly even for short and medium range reordering.

%%%%%%%%%%%%%%%%%%%%%%%%%%%%%%%%%%%%%%%%%%%%%%%%
% Others
%%%%%%%%%%%%%%%%%%%%%%%%%%%%%%%%%%%%%%%%%%%%%%%%
\subsection{Other SMT Models}

%%%%%%%%%%%%%%%%%%%%%%%%%%%%%%%%%%%%%%%%%%%%%%%%
% Hiero
%%%%%%%%%%%%%%%%%%%%%%%%%%%%%%%%%%%%%%%%%%%%%%%%
\subsubsection{Hierarchical Phrase-Based SMT}
Hierarchical phrase-based machine translation (aka \emph{hiero} or HPB-SMT) extends the phrase-based models concept of phrase from naive contiguous words to a sequence of words and sub-phrases \citep{chiang2005hierarchical}. Within the hiero model, translation rules are created using make use of the phrases extracted from the PB-SMT and the reordering of the subphrases. For example, \emph{`She likes him"} is translated to German as \emph{``Er gef\"{a}llt ihr"}\footnote{\emph{Er} is the third-person singular masculine pronoun, while \emph{ihr} is the third-person singular feminine pronoun}; this reordering can be expressed as a lexicalized \emph{gappy} synchronous hierarchical rule using \emph{X$_{1}$} and \emph{X$_{2}$} as placeholders for  subphrases.


\begin{equation}
< X{_1}\ likes\ X{_2} ,\ X{_2}\ gefaellt\  X{_1} >
\end{equation}

Hierarchical phrase-based models can also model discontiguous phrases such as the long distance dependence between the verb and its negation in German, e.g.

\begin{equation}
< X{_1}\ do\ not\ like\ X{_2}\ ,\ X{_2}\ gefaellt\  X{_1}\ nicht>
\end{equation}

In the hiero model, these translation rules are the production rules of a Synchronous Context-Free Grammar (SCFG). Target language translations are generated from both the SCFG parses and the surface string inputs. The translation rules are induced from a parallel corpus without linguistic annotation from any grammatical formalisms. The induction is based on distributional properties of the PB-SMT style phrases. Arguably, SCFG provides minimal subphrasal lexical selection syntax that is agnostic to any linguistic commitments or assumptions. 

Although the hiero model provides more robust translation possibilities, the size of the grammar is exponential because of the arbitrary re-orderings between the source and target language. \cite{zhang2006synchronous} introduced a linear-time algorithm for factoring syntactic re-orderings by restricting synchronous rules on the source-side grammar to binary branching nodes when possible. The binarization of the SCFG rules significantly improves the speed and accuracy of the hiero models. 
% Alternatively, \cite{huang2007binarization} binarized the target side grammar and similar results to source side SCFG binarization. 

Most open-sourced implementations of machine translation decoders support both phrased-based and binarized SCFG hiero models, they differ primarily by the programming language used in the implementation and the varying computing structures used to parse the trees and the formalisms used to present the same phrased-based and hiero model \citep{koehn2007moses,hoang2008design,li2009joshua,weese2011joshua,cdec}.

Still, neither the basic phrase-based model nor the basic hierarchical phrase-based model incorporate any linguistic structure such as syntax, morphology, or semantics beyond surface strings.

%%%%%%%%%%%%%%%%%%%%%%%%%%%%%%%%%%%%%%%%%%%%%%%%
% Factored SMT
%%%%%%%%%%%%%%%%%%%%%%%%%%%%%%%%%%%%%%%%%%%%%%%%

\subsubsection{Factored SMT}

In the early days of SMT, the importance of linguistic information to translation was recognized \citep{brown1993mathematics}:

\blockquote{\emph{But it is not our intention to ignore linguistics, neither to replace it. Rather, we hope to enfold it in the embrace of a secure probabilistic framework so that the two together may draw strength from one another and guide us to better natural language processing systems in general and to better machine translation systems in particular.}}

Factored SMT embarked on the task of effectively incorporating linguistic information from taggers, parses and morphological analyzers into the machine translation pipeline. It is motivated by fact that (i) linguistic information provides a layer of disambiguation to address the ambiguity of natural language, (ii) generalized translation of out-of-vocabulary (OOV) words can overcome sparsity of training data and (iii) arbitrary limits are replaced with linguistic constraints put in place in the decoding process too keep the search space tractable \citep{Hoang09aunified,koehn2010moreannotations,hoang2011improving}.

The factored model extends the phrase-based model by reformulating each word as a vector of factors and each input factor produces an equivalent output factor in the target language. For example, a vector of a source word can be made up of its surface form lemma, POS tags and morphological annotations decoded into a vector in the target language. Then the surface form of the target language word is generated given the decoded vector. The decoding process to find the most appropriate translation follows the log-linear model as in the phrase-based model where the translation step and the generation step are regarded as additional components to the log-linear model (in addition to the existing language model and reordering model in phrase-based MT) \citep{koehn2007factored}. 

The integration of morphological information remains an impetus for deploying factored based models when (i) translating from morphologically rich languages (that cause model sparsity) due to under-sampled morphological variants of the same word \citep{bojar2007english} or (ii) translating from morphological poor languages (that causes spurious ambiguities) to morphological rich languages that requires word declensions and conjugations that are lacking in the source language \citep{ramanathan2009case}.


\subsubsection{Syntax-Based SMT}

Early work on syntactically informed\footnote{ ``Syntactically informed'' refers to the linguistic theory of syntax.} SMT developed in tandem with the various components of phrase-based SMT (i.e. reordering, distortion, hiero, etc.). Based on the finite state automata, \cite{wu1997stochastic} and \cite{wu1998machine} introduced the notion of Inversion Transduction Grammars (ITG) and Stochastic Bracketing Transduction Grammars (SBTG) where every terminal symbol (word/phrase) is marked for two output streams, (i) the non-terminal node to parse upwards towards the top of the parse tree and (ii) the equivalent terminal node on other language; the non-terminal nodes are represented as classes of derivable substring pairs. 

\cite{yamada2001syntax} presented a syntax-based SMT model that transforms a source language parse tree to its target language counterpart by applying stochastic operations at each node. By doing so, they can exploit the rich parsing resources for  English. In their approach, they flattened trees to allow more reordering possibilities. The decoding process for their syntax-based model emulates a bottom-up parsing problem where nodes are translated individually and hypothesis pruning and hypothesis combinations are applied when the parser goes towards the top of the tree. 

The stochastic operations proposed in \citeauthor{yamada2001syntax} were formalized as a theory to automatically derive from word-aligned corpora a minimal set of syntactically motivated transformation rules \citep{ghkm2004}. These transformation rules (aka \emph{GHKM rules}) map the input string (words/phrases) to the output tree fragments and \cite{galley2006scalable} showed that learning the probabilities of the rules with the EM algorithm produce ``contextually-richer tree" (i.e. SCFG rules with more nodes). Likewise, syntactic parses and the surface string input can be feed into the decoder to generate phrases \citep{Huang2006T2S} and the annotations used to generate the transformation rules do not need to be restricted to syntactically structured trees, \cite{liu2008improved} extended the model with semantic role labels. These models are usually dubbed the \emph{String2Tree} and \emph{Tree2String} SMT models.

Rather than using a single best syntactic parse, \cite{mi2008forest} showed that using a forest of n-best syntactic parses of the source sentence improves upon the 1-best tree-based translation models. \cite{neubig13travatar} maintains a working implementation of the Forest2String MT systems using tree transducers\footnote{http://www.phontron.com/travatar/}.

The idea of generating GHKM rules from annotations is not restricted to a string input or output; \emph{Tree2Tree} models can be learnt from word-aligned corpora. \cite{zhangt2t} mapped source language tree fragments to target language fragments using Synchronous Tree Substitution Grammar (S-TSG). The Tree2Tree model is able to capture non-syntactic constituent phrases and discontinous phrases using linguistically structured features, additionally, it supports stratified structure reordering of larger trees. Similarly, \cite{shieber2007probabilistic} proposed a Tree2Tree model using probabilistic Synchronous Tree Adjoining Grammar\footnote{It is worth noting that the non-probabilistic S-TAG grammar formalism precedes the ITG, SBTG and SCFG \citep{shieber1990synchronous}} (S-TAG).

\subsubsection{Dependency Models and Deep Tecto MT}
 
\cite{lin2004path} introduced path-based models that extract a set of transfer rules from the corpus and each rule transforms a path in the source language dependency tree into a target language dependency tree fragment; effectively searching for the best translation problem becomes a graphical problem to find the minimum path that covers the source language dependency tree. \cite{menezes2005dependency} proposed a shallower dependency treelet approach uses the source side dependency as a tree-based ordering model and suggested that the model can be improved by adding information such as semantic roles or morphological features. 

Like the hiero model, \cite{xiong2007dependency} extracted transfer rules using the source side dependency treelet and gappy target language string fragments. \cite{shen2008new} presented a String2Tree model that extracted gappy string fragments in the source language mapping to the target language dependency treelet; this exploits a target dependency language model that was previously unavailable for the dependency Tree2String model. 

\cite{tectomt2003} proposed a tectogrammatical model (TectoMT) that is also based on dependency trees, in addition, TectoMT includes morphological analysis and generations \citep{eisner2003learning,vzabokrtsky2008tectomt,popel2010tectomt}. Similar to how phrase-based models map smaller fragments to larger ones, \cite{bojar2008phrase} extended the TectoMT model by mapping arbitrary connected fragments of the dependency tree. 
% Chapter 2.2
\section{The Mathematics of Statistical Machine Translation}

The sections above described a zoo of SMT models with varying levels of linguistic information incorporated into the different models. In this section, we describe in detail the phrase-based SMT models that are used in this thesis. 

Let's consider the scenario of translating French text into English, to formalize the convention, we will use \emph{\textbf{s}} to denote a source language (i.e. French) sentence and \emph{\textbf{t}} to denote a target language (i.e. English) sentence. 

The objective of the MT system is to find the best translation \emph{\textbf{\^{t}}} that maximizes the translation probability p(\emph{\textbf{t}}${|}$\emph{\textbf{s}}) given a source sentence \emph{\textbf{s}}:

\begin{equation}
\hat { t } =\underset { t }{ argmax } \ p(t|s)
\end{equation}

\noindent Applying Bayes' rule, we can factorize p(\emph{\textbf{t}}${|}$\emph{\textbf{s}}) into three parts:

\begin{equation}
p(t|s)=\frac { p(t) }{ p(s) } p(s|t)
\end{equation}

\noindent Substituting our  p(\emph{\textbf{t}}${|}$\emph{\textbf{s}}) back into our search for the best translation \emph{\textbf{\^{t}}} using \emph{argmax}:

\begin{equation}
\begin{aligned}
\hat { t } &=\underset { t }{ argmax } \ p(t|s) \\
           &=\underset { t }{ argmax } \ \frac { p(t) }{ p(s) } p(s|t) \\
           &=\underset { t }{ argmax } \ p(t) p(s|t)
\end{aligned}
\end{equation}

We note that the denominator p(\emph{\textbf{s}}) can be dropped because for all translations the probability of the source sentence remains the same and the \emph{argmax} objective optimizes the probability relative to the set of possible translations given a single source sentence. 

At first glance, the formulation seems counter-intuitive in that in order to achieve the best translation in the target language \emph{\textbf{\^{t}}}, we compute the component p(\emph{\textbf{s}}${|}$\emph{\textbf{t}}). This is derived from Bayes' rule and corresponds to casting the translation problem as an instance of the noisy channel model in information theory \citep{shannon2001mathematical}.

We can explain this anecdotally, consider our machine translation system as a human translator who is a native speaker of the target language (let's say English). When he/she hears the source language sentence (i.e. French), he/she will conceive of an English sentence and we can consider the p(\emph{\textbf{t}}) component as the grammaticality of that English translation.

The translator then tries to check that he/she has achieved the best translation of the source sentence \emph{\textbf{s}} given the hypothesized English sentence, \emph{\textbf{t}}. And we can consider this process as the p(\emph{\textbf{s}}${|}$\emph{\textbf{t}}) component of our machine translation system.

\noindent \cite{brown1993mathematics} provides another anecdotal account to describe the noisy-channel model\footnote{The notation in the quotation has been modified to suit the notation used in this thesis.}:

\blockquote{\emph{As a representation of the process by which a human being translates a passage from French to English, this equation is fanciful at best. One can hardly imagine someone rifling mentally through the list of all English passages computing the product of the a priori probability of the passage, p(\emph{t}), and the conditional probability of the French passage given the English passage, p(\emph{s}${|}$\emph{t}). Instead, there is an overwhelming intuitive appeal to the idea that a translator proceeds by first understanding the French, and then expressing in English the meaning that he has thus grasped. Many people have been guided by this intuitive picture when building machine translation systems.}}

In another words, the source sentence probabilistically passes through the noisy channel  p(\emph{\textbf{t}}${|}$\emph{\textbf{s}}) to result in the target sentence:


\begin{equation}
\begin{aligned}
p(s)\quad \quad ->\quad p({ { t } }{ | }{ { s } })\quad \quad ->\quad \quad f\\ 
\quad \quad source\quad \quad \quad \quad \quad channel\quad \quad \quad \quad target
\end{aligned}
\end{equation}

We need to reason backwards to the best \emph{\textbf{\^{s}}} in the source that corresponds to \emph{\textbf{\^{t}}}. We know the source probability p(\emph{s}) and the channel probability p(\emph{\textbf{t}}${|}$\emph{\textbf{s}}); i.e. how likely is \emph{\textbf{s}} corrupted into \emph{\textbf{t}}

\subsubsection{Log-linear Models}

Extending the noisy channel model, \cite{ochneymaxentsmt} simplified the integration of additional model components using the \emph{log-linear model}. The model defines feature functions \emph{h(x)} with weights ${\lambda}$ in the following form:

\begin{equation}
P(x)=\frac { exp(\sum _{ i=1 }^{ n }{ { \lambda  }_{ i }{ h }_{ i }(x) }) }{ Z } 
\end{equation}

where the normalization constant \emph{Z} turns the numerator into a probability distribution. In the case of a simple model that contains the two primary features from the noisy channel model, we define the components as such:

\begin{equation}
\begin{aligned}
{ h }_{ 1 }(x) &= p(t) \\
{ h }_{ 2 }(x) &= p(s|t)
\end{aligned}
\end{equation}

and the \emph{h(x${_1}$)} and \emph{h(x${_2}$)} are associated with the ${\lambda}$${_1}$ and ${\lambda}$${_2}$ respectively. 

The flexibility of the log-linear model allows for additional translation feature components to be added to the model easily, e.g. adding p(\emph{POS${_s}$}${|}$\emph{POS${_t}$}) to account for the translation of the part-of-speech (POS) transfers across the source and target language. 

Additionally, the weights ${\lambda}$ associated with the \emph{\textbf{n}} components can be tuned to optimize the translation quality over the parallel sentences, \emph{\textbf{D}} (often known as the development set):

\begin{equation}
{ \lambda  }_{ 1 }^{ n }=\underset { { \lambda  }_{ 1 }^{ n } }{ argmax } \sum _{ d=1 }^{ D }{ \log { { P }_{ { \lambda  }_{ 1 }^{ n } }({ s }_{ d }|{ t }_{ d}) }  } 
\end{equation}
 
 
% Chapter 2.3
\section{Using Lexical Information in Statistical Machine Translation}

At the transition from the word-based to phrase-based SMT paradigm, \cite{tanaka2003noun} highlighted that compound nouns are a major issue in machine translation because of their low frequencies and the high productivity of noun-noun compounds. 

The noun-noun compound issue in machine translation has been particularly difficult for languages that do not separate them with whitespaces, e.g. Chinese \citep{wang2007chinese,chang2008optimizing,chang2009disambiguating,pu2015leveraging}, Japanese \citep{kitamura1996automatic,cherny2000translation,baldwin-tanaka:2004:ACLMWE,pinkham2008machine,tsuji2006automatic}, Vietnamese \citep{khanhjapanese} and German \citep{rackow1992automatic,popovic2006statistical,stymne2008german,stymne2013generation,weller2014ComAComA,cap-EtAl:2014:EACL}. It remains unsolved even with the recent advancement in character-level neural machine translation \citep{tran-bisazza-monz:2014:EMNLP2014,daiber2015machine,SennrichHB15}.

Additionally, researches have also investigated improving translations of other MWEs such as phrasal verbs \citep{simova2013improving,kordoni2014multiword,cholakovkordoni2014} and named entities \citep{hermjakob2008nametrans,Nothman2013,manawi2014}. Often these bilingual MWEs are extracted automatically using a combination of statistical and heuristics-based approaches and added as additional parallel training data prior to training the machine translation system \citep{tsvetkov2012extraction}. 

Another wealth of approaches on injecting lexical information in the SMT springs from the task of adapting machine translation systems to a specific-domain. While MWE motivated SMT research to focus on translating the MWEs correctly, domain adaptation for machine translation researches focused on clever ways to incorporate various resources (generic / in-domain parallel corpora, monolingual corpora and automatically extracted or manually crafted dictionary / terminologies) \citep{nanba2009automatic,sanchez2011integrating,arcan2014identification,minarro2015acquisition} or integrating them into Computer Assisted Translation (CAT) tools to aid human translation \citep{tezcan2011smt,skadicnvs2013application}

The following sections of the thesis briefly survey seminal research that incorporated MWEs in SMT or used various lexical resources to adapt the machine translation system to a specific domain. The train of thought remains the same in finding novel and effective ways to incorporate additional lexical information to the standard SMT pipeline.


\subsection{Dictionaries are Data too}

\emph{"Machine translation depends vitally on data in form of large bilginual corpora, [but] bilingual dictionaries are also a source of information"}. \cite{brown1993dictionaries} showed that a bilingual dictionary can be used as an additional parameter to the machine translation system and including a dictionary can improve the maximum likelihood estimates of the bilingual text alignments. 

Given a dictionary entry with the source language word/phrase \emph{\textbf{s}} and its corresponding \emph{\textbf{m}} number of translations \emph{\textbf{t${_1}$, ... t${_m}$}} and if we consider that the lexicographer has chosen the translations from a random sample of \emph{\textbf{c}} instances, the probability of translation for the dictionary entry is:

%p(\emph{t${_1}$, ... t${_m}$} ${|}$ \emph{e}) ...}

\begin{equation}
p(f_1, ..., f_m | e) = \sum _{ c }^{  }{ \sum _{ { c }_{ 1 }>0 }^{  }{ ...\sum _{ { c }_{ m }>0 }^{  }{ \left( \begin{matrix} c \\ { c }_{ 1 }...{ c }_{ m } \end{matrix} \right)  } p(c|s) }  } \prod _{ i=1 }^{ m }{ { p({ t }_{ i }|s) }_{ { c }_{ i } } } 
\end{equation}

Decrypting the equation from the right to left, the product of \emph{p(\emph{t}${|}$s)} is the usual probability of the translation given the source word/phrase and the subscript in \emph{p(\emph{t}${|}$s)}${_{c_i}}$ refers to the subset of the global word/phrase translation probabilities given the sample of example sentences that the lexicographer has chosen.

The $ [ \sum _{ { c }_{ 1 }>0 }^{  }{ ...\sum _{ { c }_{ m }>0 }^{  }{ \left( \begin{matrix} c \\ { c }_{ 1...m } \end{matrix} \right)  } p(c|s) } ] $ part of the equation can be simply thought of the ``effective multiplier" that changes the word/phrase translation probabilities given the lexicographer's choice of the sampled instances. For all possible \emph{\textbf{c}} number of examples that a lexicographer can generate for a given source word, \emph{\textbf{s}}, the lexicographer chose \emph{\textbf{m}} number of examples to explain the \emph{\textbf{s}} and \emph{\textbf{m}}  translations for \emph{\textbf{s}}. 

Simplifying the computation, \cite{brown1993dictionaries} mathematically estimated the binomial distribution as a Poisson distribution, as such:

\begin{equation}
p(f_1, ..., f_m | e) = { exp }^{ -\lambda (s) }\prod _{ i=1 }^{ m }{ { (exp }^{ -\lambda (s)p({ t }_{ i }|s) } } -1)
\end{equation}

where they introduced the $\lambda (s)$ variable to represents the Poisson distribution mean and that simplifies the translations of the word \emph{\textbf{s}} as the product of each translation. To put it in \citeauthor{brown1993dictionaries}'s (1993b) words:

\blockquote{Imagine that a lexicographer, when constructing a [source language] entry for the English word or phrase \emph{\textbf{s}}, first chooses a random size \emph{\textbf{c}}, and then selects at random a sample of \emph{\textbf{c}} instances of the use of \emph{\textbf{s}}, each with its French translation [in the target language]. We imagine, further that he includes in his entry for \emph{\textbf{s}} a list consisting of all the translations that occur at least once in his random sample. The probability that he will, in this way, obtain the list \emph{t${_1}$, ... t${_m}$} is p(\emph{t${_1}$, ... t${_m}$} ${|}$ \emph{s}) ...}

Although, \cite{brown1993dictionaries} provided a mathematical account to incorporate dictionary information into an SMT system, they did not empirically test the effects of the ``effective multiplier" in a standard machine translation task setup evaluated using Word Error Rate (WER) or other machine translation evaluation metrics. However, they did show that the effective multiplier generally has a greater effect on low frequency words (${<}$5). 

%This early work on dictionary has pre-empted the out-of-vocabulary or rare word problems that happens when the datasets use for machine increases and the Zipfian tail of hapax legomenon.

\subsection{Augmenting Manual Dictionaries to Improve Statistical Machine Translation}

\cite{Vogel04augmentingmanual} showed that a statistical machine translation system can be improved when the training data has been extended with dictionary entries and their various forms transformed using simple morphological rules (e.g. plural forms and verb inflections). They augmented Chinese-English dictionary entries with new English translations by (i) first automatically generating plural forms of the noun entries and adding the definite and indefinite determiners and generating verb forms by inflecting them with -s, -ed, -ing and  with the infinitive form `to' and (ii) then filtering out word forms generated in step (i) if they do not appear in a large monolingual English corpus.

\begin{equation}
p(s|t)=\prod _{ j }^{  }{ \sum _{ i }^{  }{ p({ s }_{ j }|{ t }_{ i }) }  } 
\end{equation}

The probabilities of each lexical entry is assigned by calculating the product over the probability of the source words \emph{\textbf{s}} and the probability of each source word \emph{\textbf{s${_j}$}} is the sum of the translation probabilities of the source word \emph{\textbf{s${_j}$}} given its \emph{\textbf{t${_i}$}}, where \emph{p(\textbf{s${_j}$} ${|}$ \textbf{t${_i}$})} is the probabilities generated from the word/phrase alignment process. However, there is no indication of how the probabilities are assigned if the lexical entry does not appear in the bilingual data which was used to generate the word/phrase alignments. Additionally, they suggested that the probability of the lexical entries can be renormalized and it might help to improve the NIST scores when only the lexicon (without bilingual corpus) is used to train the model.

Simply by adding the lexicon to the baseline system using the probabilities assigned as mentioned above, \cite{Vogel04augmentingmanual} achieved improved NIST scores and the system is further improved using the augmented dictionary. 

\cite{Vogel04augmentingmanual} leveraged on the isolating (non-inflecting) nature of Chinese (source language) which simplifies the probability assignments of the augmented dictionary without considering multi-to-multi translations since only the English translations can be augmented. However, their best attempt at adding the augmented dictionary showed statistically significant\footnote{at 95\% confidence interval \citep{zhang2004interpreting}} but minimal improvement (+0.38) to the baseline system, from 5.40 to 5.78 BLEU. This is in the scenario where they have trained the system using only the augmented dictionary and a large monolingual corpus for the language model; they did not use any parallel data other than the augmented dictionary.

Since the dictionary augmentation only increases the word/phrase alignment fertility when translating from the isolating language (Chinese) to the inflectional one (English), the evaluations were uni-directional from Chinese-English. 

\subsection{Grouping Multi-Word Expressions in Statistical Machine Translation}

As Statistical Machine Translation evolved from word to phrase based approaches, the alignment models are still inherently based on word to word inferences \citep{vogel1996hmm,och2003systematic}. \cite{lambert2006grouping} proposed the idea of grouping  Multi-Word Expressions (MWE) prior to training the SMT system in the hope that the alignments created between the source and target language text depend on linguistic constituents instead of contiguous ngrams that acknowledge ngrams with partial constituents as valid phrases to be used in the decoding process. They motivate their proposal with the following example:

%\cite{lambert2006grouping} motivated the use of Multi-Word Expressions (MWE) in SMT with the following example. 

The phrase ``fire engine" is a fixed expression that is translated to ``camion de bomberos" in Spanish, and the viterbi word-to-word alignments will probably provide the following word alignments \emph{``camion$<$-$>$truck", ``bomberos$<$-$>$firefighters", ``fugeo$<$-$>$fire" and ``maquina$<$-$>$engine"} which will not yield the ``camion debomberos" translation since there are no combinations of 1-to-1 alignments which will lead to ``camion de bomberos". 

Intuitively, the example does behoove the grouping of MWEs prior to SMT training. But they overlook the basic assumption that word alignments are performed at sentence level and sub-phrasal word-to-word alignment occurs because the phrase appears on both the source and target language; given that there are occurrences of the phrase pair in the parallel corpus, there will be probability assigned to `\emph{`bomberos$<$-$>$fire"} but the probability will be lower than \emph{``bomberos$<$-$>$firefighters"} that is a natural dictionary entry independent of the context. Moreover, given a well-built language model, the SMT should provide a low perplexity to the contiguous phrases if they are observed in the monolingual corpus. 

Furthermore, \cite{lambert2006grouping} overlooked the translation model training after the word alignment process in phrase-based machine translation which capitalizes on the alignment `\emph{consistency}' which essentially requires asymmetric alignment points to extract `consistent phrases' that are saved in the phrase table used in decoding. By pre-identifying the asymmetric alignments and single-tokenizing them, the phrase extraction algorithm has lesser information when building the phrase table. 

\cite{lambert2004alignment} introduced a novel approach to extract bilingual MWEs extraction using asymmetric word alignments between the source and target language phrases (i.e. non-exclusive one-to-one alignments between contiguous phrases). These asymmetric phrases are then scored by their minimal probability computed using their relative frequency between the source and target phrase (i.e. the $argmin$ between relative frequency of the source phrase and the target phrase). The top scoring candidates are heuristically filtered using language specific rules.

Using the automatically extracted bilingual MWEs as a dictionary, \cite{lambert2006grouping} single-tokenized all instances of those MWEs in the SMT training corpus, e.g. whitespaces within the phrase \emph{``camion de bomberos"} would be replaced by underscore \emph{``camion\_de\_ bomberos"} and the components of SMT would treat the phrase as a single token instead of a multi-token entity.

When comparing the results of the Spanish-English phrase-based model built on a corpus with single-tokenized MWEs against a baseline system trained on normal text, \cite{lambert2006grouping} reported a slight decrease in BLEU scores. They reported a drop of -3.0 BLEU (ES-EN: 54.7 to 51.7) and -0.2 BLEU  (EN-ES: 47.2 to 47.0). The significance of the difference in BLEU scores is unreported thus the effect of grouping MWEs in SMT is inconclusive. 

\subsection{Domain Adaptation for Lexically Informed Statistical Machine Translation}

\cite{Koehn2007domain} typified the domain adaptation scenario in machine translation systems where we have a vast amount of monolingual and parallel data in a generic domain but limited amount of in-domain data. The shared tasks held by the popular Workshop for Machine Translation (WMT) across the years follow the same setup to train machine translation systems using generic domain data and tune the systems using domain-specific data \citep{callison2010findings,callison2011findings,callisonburch2012findings,bojar2013findings,bojar2014findings,WMT15}.

Using the English-French portion of the Europarl corpus\footnote{37 and 34 million French and English tokens respectively} and NewsCommentary corpus\footnote{1.2 and 1.1 million French and English tokens respectively}, \cite{Koehn2007domain} trained several French to English factored translation models and language models using various combinations of the in- and out-domain data and tested them on a test set made of news commentaries:

\begin{itemize}
\item Training a single translation model and language model using only the Europarl corpus (25.11 BLEU)
\item Training a single translation model and language model using only the NewsCommentary corpus (25.88 BLEU)
\item Training a single translation model and language model using both the Europarl and NewsCommentary corpora (26.69 BLEU)
\item Training a single translation model using both the Europarl and NewsCommentary corpora and using only the NewsCommentary in the language model (27.46 BLEU)
\item Training a single translation model using both the Europarl and NewsCommentary corpora and using an interpolated language model trained with both corpora (27.12 BLEU)
\item Alternative decoding using a single translation model trained on both in- and out-domain corpora and two separate language models trained using only the Europarl corpus and only the NewsCommentary corpus (27.30 BLEU)
\item Alternative decoding using a single language model trained on the in-domain data and two translation models trained separately using only the Europarl corpus and only the NewsCommentary corpus (27.64 BLEU)
\end{itemize}

The best results was obtained using alternative path decoding \citep{birch2007ccg} from two different translation models and a single language model trained using the in-domain data. 

One possible reason for the difference in the scores achieved by alternative path decoding using two translations models (27.64 BLEU) and simply combining the data (26.69 BLEU) is the access to varied factor probabilities from different models. When the factor probabilities are computed separately the normalizing denominators are domain dependent, which will result in varying ranks in the translation probabilities of in-domain phrases. However, the +1.05 BLEU improvement was neither tested for statistical significance nor manually evaluated; it may not reflect true improvement in translation quality. 

Relatively, the simple data concatenation of the in-domain data with the generic corpus achieved a +1.58 BLEU improvement (25.11 to 25.88 to 26.69). Also, the same improvement is reflected when only the in-domain data was used to train the language model. 

Hypothetically, the concatenation of the in-domain data with general domain data could have effectively introduced domain specific words that might not have been in the generic corpus or increase the counts of the domain specific words. These findings empirically reflect the notion of ``effective multiplier" that \cite{brown1993dictionaries} theoretically proved; SMT domain adaptation can be achieved by influencing the frequencies in the in-domain lexicon.

\subsection{Training Phrase-Based SMT Models using Domain-Specific and Generic Corpora and Dictionaries}

\cite{wu2008domain} enumerated several scenarios where researchers have access to one or more of the following  resources (i) a generic parallel corpus, (ii) an in-domain dictionary, (iii) an in-domain target language corpus and (iv) an in-domain source language corpus. \cite{wu2008domain} developed a heuristics-based domain adaptation algorithm approach that checks which of the resources is/are available before training the model differently depending on the available resource(s). 

Their default assumption is that there is always a generic parallel corpus and an in-domain dictionary. \cite{wu2008domain} developed a domain adaptation algorithm for SMT that follows the following steps:

\begin{itemize}
\item[1a.] \emph{\textbf{if an in-domain target language corpus is available}}, train two separate language model using the in- and out-domain target language corpus and use the additional language model as an extra feature function in the decoding process. Using the extra language model feature, train a phrase-based system.
\item[1b.] \emph{\textbf{otherwise}} train a standard phrase-based system on the out-domain parallel texts and in-domain dictionary.
\item[2.] \emph{\textbf{if an in-domain source language corpus is available}}, translate the source language corpus with the system built in step 1, then append the source and translation to the generic parallel text and dictionary to retrain the model and evaluate the new model on a development set. Then repeat the translation of the source language corpus and retrain the model until there's no improvement made to the development set.
\end{itemize}

The recursive retraining in step 2 is similar to the   approach introduced by \cite{ueffing2007transductive}. To combine the lexical information from both the generic corpus and the in-domain dictionary, \cite{wu2008domain} created two phrase tables separately from the generic corpus and the in-domain dictionary. They experimented with four different ways to assign the probabilities of the dictionary phrase table: 

\begin{itemize}
\item[i.] \emph{\textbf{simply adding the dictionary as additional data to the generic corpus}}
\item[ii.] \emph{\textbf{using uniform probabilities}} ($1/n$, where $n$ is the no. of possible translations for a source phrase) 
\item[iii.] \emph{\textbf{using constant probabilities}} (as reported in the paper, they set the constant to 1 for their experiments)
\item[iv.] \emph{\textbf{using corpus probabilities}} when an in-domain source corpus is available\footnote{they repeatedly translated the source corpus synthetically as in step 2 of the algorithm, then estimate the probabilities using the translation probabilities of the dictionary entries using the translation probabilities from the phrase table}
\end{itemize}

When translating from Chinese to English\footnote{Tuned and tested on IWSLT 2006 development and evaluation data} using a model trained on the CLDC Chinese-English bilingual corpus\footnote{http://www.chineseldc.org/doc/CLDC-LAC-2003-004/intro.htm} (\textbf{generic corpus}) and the Basic Travel Expression Corpus (BTEC) \citep{iwslt2006} (\textbf{in-domain corpus}) and the LDC Chinese English translation Lexicon (LDC2002L27) (\textbf{generic dictionary}) and a manually created \textbf{in-domain dictionary}\footnote{They collected dictionary entries from phrase books and the dictionary was verified with a native Chinese speaker} was used as the dictionary, their baseline system achieved 13.59 BLEU and simply adding the dictionary as additional training data improved the system by +1.93 BLEU. Using uniform, constant and corpus probabilities yielded +2.41, +2.79 and +2.13 BLEU improvement respectively over the baseline. They showed that they were able to exploit the monolingual source corpus to improve translations. 

Furthermore, when they added the target language corpus, their combined system trained using the algorithm described above scored 21.75 BLEU points, a significant improvement over the baseline system trained on the generic corpus. Without the dictionary input and the transductive learning cycles, the generic corpus with the target language in-domain corpus, the model scored 17.16. 

Additionally, \cite{wu2008domain} analyzed the number of lexical entries and the out-of-vocabulary percentage of the various resources and concludes that:

\begin{itemize}
\item Using a general-domain dictionary alone (without the in-domain dictionary) can improve the system (17.16 (baseline) $->$ 19.11 BLEU)
\item Using a manually crafted in-domain dictionary alone improves the system more than using a general dictionary (21.16 BLEU)
\item Using an automatically extracted in-domain dictionary\footnote{They extracted the in-domain dictionary from BTEC using the methods described in \cite{wu2007comparative}} alone improves the system too (19.88 BLEU)
\item Combining the generic and manually-crafted in-domain dictionary improves the system further than just the in-domain dictionary alone (21.34 BLEU)
\item Combining the generic and automatically extracted in-domain dictionary improves the system but marginally (20.49)
\end{itemize}

\cite{wu2008domain} empirically tested various improvements that different resources can make to an SMT system and showed significant BLEU improvements over their baseline system. They have also tried various ways to incorporate lexical information from manually and automatically extracted dictionaries into an SMT system. 

However, the evaluation data and the host of resources they have used in their experiments are not openly available, preventing other researchers to verify and improve upon their work. 

%It also leads to the current advancement in exploring how and whether generic and domain-specific lexicon can improve state-of-art SMT systems.

\subsection{Improving Statistical Machine Translation Using Domain-Specific Bilingual MWEs}

Motivated by the research reported in \cite{lambert2006grouping} and \cite{wu2008domain} work on using MWEs in SMT and in \cite{Koehn2007domain} on  improvements made to domain-adapted MT, \cite{ren2009improving} experimented with three different ways of integrating lexical knowledge from MWEs into SMT; viz 

\begin{itemize}
\item[i.] \emph{\textbf{Adding a bilingual MWE dictionary as additional data before training the SMT models}} ({\tt BiMWE)}
\item[ii.] \emph{\textbf{Using a binary feature function that uses 1 to represent the existence of at least one MWE translation}} matching entries in the MWE dictionary and
\item[iii.] \emph{\textbf{Adding the MWE dictionary as an additional phrase table and assigning all feature probabilities to 1}} for alternative path decoding. 
\end{itemize}

They evaluated the methods using two Chinese-English domain-specific patent corpora, one in the Traditional Chinese Medicine (TCM) domain and the other in the chemical industry domain. The training corpora were of similar sizes with 120,000 sentences that contains around 4.5 tokens for each sentence. The models were tuned and tested on a development and test set that comprised 1,000 sentences (30-40,000 words in total). 

All three methods of integrating MWE attained marginal improvement ($<$ +1.0 BLEU) over their baseline phrase-based system that achieved 26.58 BLEU in the TCM domain. 

In their error analysis, they manually evaluated the automatically extracted MWEs and found that only 76.69\% are correct. They introduce a Log-Likelihood Ratio (LLR) scoring mechanism with Gaussian priors to filter the noisily extracted MWE list and retrained their MT model with {\tt BiMWE} and found that the system improved from 26.61 to 27.15 BLEU. However, it is still marginally better than their baseline system without MWEs. Similarly, all three MWE integration methods with the filtering mechanism had marginal gains ($<$ +0.5 BLEU) over their baseline (18.82 BLEU) in the chemical industry domain.

\cite{ren2009improving} concluded their experiments with "to our disappointment, however, none of these improvements are statistically significant". 

\begin{landscape}
\newpage
\begin{table}
\centering
{\tablinesep=2ex\tabcolsep=5pt
    \begin{tabular}{lc|ccc|cc|ccc|c}
    \textbf{Previous Work}       & ~      & \textbf{+Dict}                & \textbf{+MWE}                 & \textbf{Single-tok}           & \textbf{In-Domain}            & \textbf{Domain Adapt.}        & \textbf{Passive}              & \textbf{Intrusive}            & \textbf{Pervasive}            & \textbf{+BLEU (sig.)}         \\ \hline
    
    Vogel \& Monson     & (2004) & \checkmark  & ~                    & ~                    & ~                    & \checkmark  & \checkmark  & ~                    & ~                    & \checkmark  \\
    Lambert \& Branch   & (2006) & ~                    & \checkmark  & \checkmark  & ~                    & ~                    & ~                    & ~                    & ~                    & ~                    \\
    Koehn \& Schroeder  & (2007) & \checkmark  & ~                    & ~                    & ~                    & \checkmark  & \checkmark  & \checkmark  & ~                    & \checkmark  \\
    Wu et al.           & (2008) & \checkmark  & ~                    & ~                    & ~                    & \checkmark  & \checkmark  & ~                    & ~                    & \checkmark  \\
    Ren et al.          & (2009) & \checkmark  & \checkmark  & ~                    & ~                    & \checkmark  & \checkmark  & \checkmark  & ~                    & ~                    \\
    Pal et al.          & (2010) & ~                    & \checkmark  & ~                    & ~                    & ~                    & \checkmark  & ~                    & ~                    & \checkmark  \\
    Tsvetkov \& Wintner & (2012) & ~                    & \checkmark  & ~                    & \checkmark  & ~                    & \checkmark  & ~                    & ~                    & \checkmark  \\
    Skadins et al.      & (2013) & \checkmark  & ~                    & ~                    & ~                    & \checkmark  & \checkmark  & \checkmark  & ~                    & \checkmark  \\
    Simova \& Kordoni   & (2013) & ~                    & \checkmark  & \checkmark  & \checkmark  & ~                    & \checkmark  & ~                    & ~                    & \checkmark  \\
    Meng et al.         & (2014) & \checkmark  & ~                    & ~                    & ~                    & \checkmark  & \checkmark  & \checkmark  & \checkmark  & \checkmark  \\
    Tan and Pal         & (2014) & \checkmark  & \checkmark  & ~                    & ~                    & ~                    & \checkmark  & ~                    & ~                    & ~                    \\
    Hellrich \& Hahn    & (2015) & \checkmark  & ~                    & ~                    & ~                    & \checkmark  & \checkmark  & ~                    & ~                    & ~                    \\
    Tan et al.          & (2015) & \checkmark  & ~                    & ~                    & ~                    & ~                    & \checkmark  & ~                    & \checkmark  & \checkmark  \\
    \textbf{This Thesis }        & (2016) & \checkmark  & \checkmark  & \checkmark  & \checkmark  & ~                    & \checkmark  & ~                    & \checkmark  & \checkmark  \\
    \end{tabular}
}
\caption{A Comparison of Previous Work in Integrating Lexical Resources in Statistical Machine Translation}
\label{table:dictmt}
\end{table}
\end{landscape}

\newpage
\subsection{A Overview of Integrating Lexical Information in SMT}

Table 2.1 presents an overview of the state of art in integrating lexical information in statistical machine translation. Based on BLEU score evaluation, the results shows that it is possible to achieve statistically significant BLEU gains albeit generally only a minor increment in absolute BLEU scores. 

The \textbf{Passive} column refers to the integration lexical information by adding additional training data prior to the model training. The joint \textbf{+Dict} and \textbf{Passive} columns show that the most common approach to integrating lexical information is the passive addition of manually crafted or automatically extracted parallel dictionary or terminology to the start of statistical machine translation training process \citep{Vogel04augmentingmanual,koehn2007experiments,wu2008domain,ren2009improving,skadicnvs2013application,meng2014,manawi2014,minarro2015acquisition,pervasive2015}. In some studies, they also added automatically extracted Multi-Word Expressions (MWEs) to training data prior to the training process \citep{ren2009improving,manawi2014}, indicated by the joint  \textbf{+Dict}, \textbf{+MWE}, \textbf{Passive} columns in Table 2.1.

There were also studies that explores the effects of solely adding parallel MWEs \citep{lambert2006grouping,pal2010handling,tsvetkov2012extraction,simova2013improving,kordoni2014multiword}, represented by the joint \textbf{+MWE} and \textbf{Passive} columns. This is in line with the \citep{brown1993dictionaries} hypothesis of increasing the relevant ``effective multiplier" of MWEs by increasing their frequencies in the training data. In exploring effectively multiplying the frequencies of MWEs, \cite{lambert2006grouping} experimented with single tokenizing the MWEs to trick the SMT system to treat the MWEs as a single token when extracting and decoding the ngrams; they achieved negative results from their experiments (details in Section 2.3.3). 

However, as shown by the joint \textbf{+MWE}, \textbf{Single-tok} and \textbf{Passive} columns, \cite{tsvetkov2012extraction} and \cite{kordoni2014multiword} repeated similar experiments on a different datasets and found that single tokenizing MWEs provides BLEU gains to SMT systems. Like in some other lexical information adding studies, they achieved statistical significant BLEU increments with little absolute BLEU gains. Focusing on only specific types of MWEs, e.g. named entities, verb compounds or phrasal verbs, \cite{pal2010handling} and \cite{simova2013improving} showed positive results in single tokenizing MWEs to improve statistical machine translation.

Previously research on exploiting additional lexical information primarily targets the task of domain adaptation where external resources from a different domain are included to the training data to scale the machine translation system from one domain to another. In doing so, addition lexical resources from different domains are needed. Alternatively, \cite{tsvetkov2012extraction} and \cite{kordoni2014multiword} attempted to inject domain-specificity \textbf{without using external resources from another daomain}. This is usually done by capitalizing on the lexical distribution extracted from the training data. The machine translation improvements made in this thesis follow the same train of thought where that lexical resource comes from the same domain as the training data. 

The \textbf{Domain Adapt.} column in Table 2.1 indicates the work focused on adapting machine translation from one domain to another while the \textbf{In-Domain} column indicates the researches focusing on improving domain specificity in machine translation using the resources extracted from the training data.


Other than passively adding lexical information to the training data, previous studies have attempted various ways of injecting lexical information in the various steps in the statistical machine translation training processing. \cite{koehn2007experiments} introduced the idea of using more than a single pre-trained translation model and language model that can be used as domain adaptation. In that sense, the lexical information is added not only by \textit{passively} adding them prior to the training process affecting the translation model and the language model in a monolithic one-off manner. Rather, introducing additional alternate decoding paths with the domain specific parallel corpus and/or dictionary, the lexical information addition becomes an intrusive injection to the SMT training process. We denote such ``injection'' as \textbf{Intrusive} on Table 2.1\footnote{Although the intrusive addition of lexical information remains an active field in SMT, it would not be covered beyond this overview under the limited scope of the thesis.}.

Similarly, framing the additional lexical information for domain adaptation, \cite{wu2008domain} experimented with various alternate decoding with a permutation of in-domain and out-domain language models and translation models. Furthermore, to isolate the dictionary from the in-domain parallel corpus, they create phrase tables solely from the in-domain and generic dictionaries separately and jointly before injecting the additional translation model into the alternate decoding step. They reported positive results from several experiment setups (See Section 2.3.5). Injecting lexical information from a different approach, \cite{skadicnvs2013application} added a dense binary feature to indicate the existence of an in-domain term in the SMT training process, the simple yet effective feature showed statistically significant +6.0 BLEU improvements but there was no documentation of the absolute BLEU gains and the dataset used in their experiments. Likewise, \cite{meng2014} proposed a term disambiguation, term consistency and term bracketing features that attempted to improve the SMT decoding process. But they achieved marginal and statistically insignificant BLEU increments over the baseline models. 

The last genre of lexical information addition to statistical machine translation comes from the last step of the SMT search process (aka decoding). Since the decisions made using the additional lexical information at the decoding stage would be finalized in the machine translated output, we denote such lexical information addition techniques at the decoding step as \textbf{Pervasive}. From the literature, there was only one previous work that enforces specific knowledge in the decoding process; when decoding, \cite{meng2014} whenever a hypothesis just translates a source term in a possible target term, they check whether the translation exists in a bilingual term bank. Different from the encoding the existence of a recognized in-domain term as a binary feature and allowing the log-linear decoding to decide the best possible decoding path like in \cite{skadicnvs2013application}, the pervasive technique used in \cite{meng2014} ensures translation consistency of a specific term.\footnote{This is often referred to as ``forced decoding'' and can be easily enabled using the XML decoding feature using the Moses Machine Translation Toolkit}\footnote{We note that during our interaction with the industry partners of the EXPERT project, they have informed us that these pervasive techniques are essential when delivering high quality humanly post-edited translations using the outputs from machine translation systems. We note that in the case of commercial applications, there is no little or no reference translation to determine the BLEU score of the machine translation outputs and the only measure of ``goodness'' of translation comes from the satisfaction given by the clients of the commercial companies. The Volvo incident (Section 2.4.2) reiterates the necessity to  look into these pervasive lexical information addition techniques and more importantly gain insights how the research community should change our notion of machine translation evaluation to suit actual industry needs.}

To make a comparison between \textit{passive} and \textit{pervasive} addition of lexical information, %in \cite{pervasive2015}, 
we empirically investigate the effects of both approaches on the same dataset and provide further insights on how lexical information can be reinforced in statistical machine translation. The extension of this work is described in Chapter 5, where we compare the coverage of this work in the experiments on using additional lexical information in statistical machine translation.




% Chapter 2.4
\input{background/term-onto}
% Chapter 2.4.1
\subsection{A Survey of Term Extraction Techniques} \label{sec:terminology}

A \textbf{term} is the \emph{designation of a defined concept in a special language by a linguistic expression}; a term may consist of one or more words. A \textbf{terminology} refers to the set of terms representing the system of concepts of a particular subject field (ISO 1087). The International Organization of Standardization (ISO) history of terminology traces back to \citeauthor{wuster1969}'s \citeyearpar{wuster1969} seminal article on \emph{Die vier Dimensionen der Terminologiearbeit}\footnote{The Four Dimensions of Terminological Work} which the ISO Technical Committee 37 (ISO/TC 37) builds upon in providing the common standards related to terminology work. 

A later formulation states that a term is \emph{any conventional symbol representing a concept defined in a subject field}; a terminology is the aggregate of terms, which represent the system of concepts of an individual subject field \citep{felber1984}. The core characteristic of a term is defined as \textbf{termhood}, i.e. \emph{the degree to which a linguistic unit is related to a domain-specific context} \citep{kageura1996}. In the case of multi-token terms, additional substantiation is necessary to check its \textbf{unithood}, i.e. \emph{the degree of strength or stability of syntagmatic combinations and collocations} \citep{kageura1996}.

Single token terms can be perceived as a specialized vocabulary\footnote{aka. domain-specific vocabulary} that is used specifically in a domain. The surface word representing the single token term is often polysemous and the usage of the term within a specialized domain may narrow down the set of possible senses or single out a disambiguated sense of the word. For example, the term ``\emph{classifier}" can refer to:

\begin{itemize}[nosep]
\item[(i)] a morpheme used to indicate the semantic class to which the counted item belongs, or 
\item[(ii)] a pre-trained model to identify/distinguish different classes within a dataset. 
\end{itemize}
The first definition is mainly used within linguistic research, the second within the machine learning domain. However, when \textit{"classifier"} is used in computational linguistics, its usage is ambiguous. The latter definition of classifier tends to be used more often than the former. 

In English, terms are more often multi-word expressions (MWE), primarily nominal phrases, made up of a head noun and its complement adjective(s), prepositional clause(s), or compounding noun(s). Commonly, a complex term can be analysed in terms of a head with one or more modifiers \citep{hippisley2005head}.

\subsubsection{Rule-based Term Extraction}

The linguistic properties of a term can be characterized by its syntactic context. Previous approaches to term extraction use these linguistics properties in form of Part-Of-Speech (POS) patterns. For example, \cite{justeson1995} and \cite{daille1996} used the following POS patterns to extract nominal phrasal terms:
\begin{itemize}
\item[] \textbf{EN}: ((Adj$\vert$(Noun)+$\vert$((Adj$\vert$Noun)*(NounPrep)?)(Adj$\vert$Noun)*)Noun$_{head}$
\item[] \textbf{FR}: Noun$_{head}$ (Adj$\vert$(Prep(Det)?)?Noun $\vert$V$_{inf}$ )
\end{itemize}
In the case of English, the compulsory head noun is in the final position preceded by its modifiers whereas in French, it is in the first position followed by its modifiers. The multi-word nature of Romance languages produces more terminological phrases, whereas for Germanic languages, the compounding nature of nouns derives more single token lexicalized terms. For example, an equivalent POS pattern for German would have to consist of a combination of POS and morphemic pattern:

\begin{itemize}
\item[] \textbf{DE}: ((AC$\vert$NC)+$\vert$((AC$\vert$NC)*(Noun$\vert$Prep)?)(AC$\vert$NC)*)Noun$_{head}$
\end{itemize}

\noindent Similar to the (Adj$\vert$Noun) pattern in English, the German (AC$\vert$NC) pattern is a combination of adjective/noun with occasional connective morpheme where a connective morpheme might be necessary to join the adjacent adjectives/nouns. For example, in the German compound noun \emph{Mausefalle} (\emph{Mousetrap}), the \emph{Falle} (\emph{trap}) is head noun in the final position and the word \emph{Maus} (\emph{mouse}) attaches to the head noun with the \emph{-e-} connective morpheme between the nouns.

Linguistic patterns such as these are usually used as filters to generate a list of multi-word
terms of high unithood followed by further statistical measures to re-rank or reduce the list \citep[e.g.][]{bourigault1996lexter}. \cite{frantzi2000automatic} differentiated two types of filters, viz. a close filter is strict about which strings it permits and an open filter allows more strings in the POS patterns. For example, the English pattern is an open filter that allows a wider range of multi-word term candidates than simply using a /Noun+/ that only allows delexicalized compounding nouns. 

Although state-of-art term extraction systems do not solely rely on linguistic patterns, the pattern templates are used as filters to remove candidate terms from the system output \citep[e.g.][]{zhang2004comparison,gomez2009parallel}.


\subsubsection{Statistical Term Extraction}

The basis of all statistical properties in multi-word term extraction relies on the frequency of a token or an n-gram in a corpus. Frequency counts are combined to compute co-occurrence measures (aka. word/lexical association measures) that quantify the probabilistic occurrence of a word with its neighbouring words. Cooccurrence measures are used to estimate the propensity for words occurring together.

Psycholinguistic evidence shows that word association norms can be measured as a subject’s responses to words when preceded by associated words \citep{palermo-jenkins} and humans respond quicker in the case of highly associated words within the same domain \citep{church1990word}.

Common co-occurrence measures, e.g. Dice coefficient, Mutual Information (MI), Pointwise Mutual Information (PMI), Log-Likelihood Ratio (LLR) and Phi-square ($\phi$$^{2}$) rely on three types of frequency information; (i) the frequency of a word occurring in the corpus, (ii) the joint frequency of a word occurring with another word, (iii) the total number of words in the corpus. Formally we describe them as follows:

\begin{itemize}[noitemsep]
\item[] Let \emph{f$_{i}$} be the frequency of the occurrence of a word, $i$
\item[] Let \emph{f$_{j}$} be the frequency of the occurrence of another word, $j$
\item[] Let \emph{f$_{ij}$} be the frequency of the word $i$ and $j$ occurring simultaneously
\item[] Let \emph{f$_{ij’}$} be the frequency of the word $i$ occurring in the absence of $j$
\item[] Let \emph{f$_{i’j}$} be the frequency of the word $j$ occurring in the absence of $i$
\item[] Let \emph{f$_{i’j’}$} be the frequency of both words $i$ and $j$ not occurring
\item[] Let $N$ be the size of the corpus
\end{itemize}

We further simplify the notion by having $a$ = \emph{f$_{ij}$} , $b$ = \emph{f$_{ij’}$} , $c$ = \emph{f$_{i’j}$} and $d$ =\emph{ f$_{i’j’}$}.

These basic statistical properties of word co-occurrence are combined in various ways to form more complex co-occurrence measures. The common co-occurrence measures are defined as follows:

\begin{eqnarray} 
Dice(i,j) & = & \frac { 2*a }{ ({ f }_{ i }+{ f }_{ j }) }  \\[1em]
PMI(i,j) & = & \log { a } -(\log { { f }_{ i } } +\log { { f }_{ j }) }  \\ [1em]
MI(i,j) & = & \log { a } -\log { (a+b) } -\log { (a+c) }  \\ [1em]
LLR(i,j) & = & a*\log { a } +b*\log { b } +c*\log { c } +d*\log { d }  \nonumber\\  
&  & -(a+b)*\log { (a+b) } -(a+c)*\log { (a+c) }  \nonumber\\  
&  & -(b+d)*\log { (b+d) } -(b+d)*\log { (c+d) }  \nonumber\\  
&  & +(a+b+c+d)*\log { (a+b+c+d) } 
\end{eqnarray}
\begin{eqnarray} 
{ \phi  }^{ 2 } & = & \frac { (a*d-b*c)^{ 2 } }{ (a+b)(a+c)(b+c)(b+d) }  
\end{eqnarray} 


Distributional properties can be viewed as localized statistical properties. The statistical properties in the previous section make use of global count occurrences of words to extract co-occurrence statistics between words. The distributional properties relate to (i) the number of documents that a word occurs within a corpus and/or (ii) the differing counts of a word occurring across two or more corpora.

A common measure is the term frequency – inverse document frequency (\emph{tf-idf}). The term frequency reflects the global counts of a word and the inverse documen frequency measures the spread of the word throughout the document collection. Formally,

\begin{itemize}[noitemsep]
\item[] Let \emph{f$_i$} be the no. of times a term occurs in all documents
\item[] Let \emph{n$_i$} be the no. of documents where the term $i$ occurs
\item[] Let \emph{N$_{doc}$} be the total no. of documents in a corpus
\item[] The term frequency: $tf$ = \emph{f$_i$}
\item[] The document frequency: $df$ = \emph{n$_i$} / \emph{N$_{doc}$}
\item[] In logarithmic space, the inverse document frequency: \emph{idf} = \emph{log}(\emph{N$_{doc}$} / \emph{n$_i$})
\item[] And, \emph{tf-idf} = $tf * idf$
\end{itemize}

A high word frequency might favor the global statistical co-occurrence measure however if the mass of the counts comes from a low number of documents, it will reflect a low tf-idf score deeming the term to be document-specific.

Other than using the distributional properties of words within a corpus, it is also helpful to compare the distribution of words across corpora. By comparing a domain specific corpus distribution to a general corpus, we can determine the weirdness of ratio of term frequencies across the corpora \citep{ahmad1999university}. The weirder a term, the more domain-specific a term is and the more likely it is to be a term candidate to form the terminology of a specific domain. We can simply refer to the relative frequency ratio across the corpora as such:

\begin{equation}
weirdness(i)==\frac { { f }_{ i }^{ D }*{ N }_{  }^{ G } }{ { f }_{ i }^{ G }*{ N }_{  }^{ D } } 
\end{equation}

where ${ f }_{ i }^{ D }$ is the frequency of a term $i$ in a domain-specific corpus and ${ f }_{ i }^{ G }$ is the frequency of the same term in a generic corpus;  ${N }_{  }^{ G }$ and ${N }_{  }^{ G }$ is the total number of tokens in the generic corpus and a domain specific corpus.

\cite{frantzi1998c,frantzi2000automatic} introduced a method to use both linguistic and statistical information using C-value and NC-value. They start with a set of POS patterns and a stop word list to pre-filter possible n-grams before they calculate the n-gram’s termhood using the C-value metric and the concept of nested terms. Nested terms refer to those terms that appear within other longer terms and may or may not appear by themselves in the corpus \citep{frantzi1998c}, e.g. ‘floating point’ is a nested term because it is also found in ‘floating point arithmetic’.

For non-nested terms, the C-value accounts for the length of the term candidate and its frequency. For nested terms, the C-value subtracts the average number of times the term is nested in other term n-grams. Thus if ‘floating point’ occurs as a nested term candidate as often or more than it does as an independent term, then it will have low C-value.  Formally:

\begin{itemize}[noitemsep]
\item[] Let $NG$ be the set of all n-grams possible from a corpus.
\item[] Let $T$ be the set of all n-grams possible after using a POS pattern filter such that $T\subset NG$
\item[] Let $t$ be a candidate term that is filtered from the full list of n-grams, and
\item[] Let \emph{T$_N$} be the set of terms that contains nested terms with t such that $t\subset T_N \subset T$
\end{itemize}

Given the definition of \emph{T$_N$}, we calculate the \emph{C-Value} of a term $t$ as follows:

\begin{equation}
C-value(t)=
	\left \{
		\begin{matrix} 
		\log { |t| } *{ f }_{ t }  & if \ t \ is \ nested\\
		\log { |t| } *{ (f }_{ t }-{ 1 }/{ { f }_{ { t }_{ N } } }*
			\sum _{ i\in { T }_{ N } }^{  }{ { f }_{ i } }  & otherwise
		\end{matrix} 
	\right .
\end{equation}

From the C-Value equation, the C-value will be high for long non-nested strings with high frequency. The limitation of the C-value is that it can only be applied to multi-word terms.

And extension of the C-value is the NC-value which accounts for the context in which the term occurs. The NC-value re-ranks the term candidates extracted from the C-values by looking into the previous words occurring before the term. This is motivated by the notion of extended terms, where the terms constrain the modifiers they accept \citep{sager1980english}. This contextual constraint manifests itself as a weight to account for the number of nested terms within in the candidate term; it is then normalized by the cumulative context weight (CCW). Formally it is defined as follows:

\begin{equation}
NC-value(t)=0.8*C-value(t)+0.2\sum _{ c }^{ { C }_{ t } }{ f(c|t) } 
\end{equation}

where

\begin{itemize}[noitemsep]
\item[] $C-value(t)$ is the C-value of $t$
\item[] C is the set of distinct context words of $t$
\item[] $f(t|c)$ is the conditional probability of $t$ given that $c$ occurs within the context 
\end{itemize}

\citeauthor{frantzi2000automatic}'s \citeyear{frantzi2000automatic} notion the $f(t|c)$ probability differs from the common Bayes rule derivation of $f(t,c)/f(c)$, instead, the conditional probability here is calculated by taking $f(t,c) * c_t / n_t $, where $c_t$ is the no. of terms that have the context word $c$ and $n_t$ is the no. of total terms extracted and filtered from the corpus.

C-values and NC-values have proved to perform well \citep{zhang2008comparative,lossio2013combining} (Zhang et al. 2008,). However it does not measure termhood as defined by \cite{kageura1996}. The formulation of the NC-value measures how consistently a phrase can be a term but it does not exactly contribute to select any n-grams to be a term. Inherently, the term candidate selection is handled by the POS pattern filter and the NC-value reranks the terms to further threshold the list of candidates. Although it was a solution created close to a decade ago, it is still a common algorithm used for commercial and academic term extraction\footnote{https://code.google.com/p/jatetoolkit/wiki/JATEIntro}


% Chapter 2.4.2
\newpage
\subsection{The Importance of Terminology in Machine Translation}

In today`s globalized world, the ability to localize information into a foreign market is crucial to business expansion and machine translation (MT) is the an important means to help translate the sheer amount of information that global businesses need to process daily.

Businesses benefit from translating documents automatically by accelerating corporate communication and an MT system sensitive to domain-specific terminology is crucial in ensuring uniform and clear corporate language \citep{porsiel2011}. Yet ``\emph{terminology is the biggest factor in poor translation quality}" \citep{warburton2005} and ``\emph{businesses often fail to see terminology management as a way to cut costs}" \citep{clientsidenews2006}. \cite{lionbridge2010} reported, ``\emph{approximately 15 percent of all globalization project costs arise from rework, and the primary cause of rework is inconsistent terminology".}

The use of MT is worthwhile if the following prerequisites are given: there must be a specific corporate terminology in the largest possible scope and of the best possible quality in both the source and target languages \citep{porsiel2008}.  As highlighted, the two main points in ensuring that terminology is useful for improving MT requires (i) the largest possible scope and of best possible quality, i.e. \emph{recall and precision} and (ii) in both source and target languages, i.e. \emph{bilingual}.  

In 2008, the Swedish car manufacturer Volvo was found partly to blame for a car accident which killed two school children. The expert engineer appointed by the court criticized the poor translation of a Third-Party Intermediary in the car manual on power-assisted brakes, the expert engineer stated, `\emph{... there is some cause for highlighting the fact ... that the technical product information EWP S 2631 (D 710) issued by the manufacturer and translated by the import was imprecise and poorly written}`. Volvo was fined 200,000 Euros for involuntary manslaughter and bodily injury (c.f. \citealp{hoffmeister2014}). This called for a global concerns when handling terminology translations in technical manuals, especially in current translation workflows that incorporate machine translation with human post-editing. 


% Chapter 2.4.3
\newpage
\subsection{A Survey of Ontology Induction Techniques} \label{sec:ontology}

Aristole's metaphysical categorization of worldly concepts\footnote{Aristotle, \emph{Metaphysics}, I, 4, 985} attempted to classify concepts into a universal hierarchical structure; later known as an ontology. Ontologies are often perceived as a hierarchical organization of concepts in a tree-like structure where: 

\vspace{2mm}
\begin{itemize}[nosep]
\item \emph{\textbf{Concepts}} are the atomic units that relate to each other within the taxonomy
\item \emph{\textbf{Relations}} are the links that bind the concepts 
\item \emph{\textbf{Root}} is a top most concept on the hierarchy
\item \emph{\textbf{Instances}} are particular referents/instantiations of a concept.
\end{itemize}

\vspace{2mm}
\noindent For instance, \emph{dog}, \emph{mammals} and \emph{animals} are concepts within a taxnomy. They would be organized as follows within a taxonomy:

\vspace{2mm}
\hspace{55mm} $\top$ $\to$ \emph{animals} $\to$ \emph{mammals} $\to$ \emph{dog}
\vspace{2mm}

\noindent The $\top$ symbol preceding the first concept indicates the \emph{root} of the ontology. And the $\to$ symbols specify the relations\footnote{in this case, a \emph{hypernym} $\to$ \emph{hyponym} relation} between the connecting concepts. As an instantiation of the lowest level concepts, one could refer to it as a referent, e.g \emph{Odie}, the dog from the \emph{Garfield} comics would be a referent, i.e. an instantiation of the \emph{dog} concept.

Traditionally, broad-coverage semantic taxonomies such as CYC \citep{lenat1995cyc}, SUMO \citep{PeaseEtAl2002sumo} have been manually created with much effort and yet they suffer from coverage limitations. This motivated the move towards unsupervised approaches for ontology induction and knowledge extraction \citep{lin2001discovery,snow2006semantic,velardi2013ontolearn}. 

Ontological induction approaches can be broadly categorized as (i) pattern/rule based, (ii) clustering based, (iii) graph based and (iv) vector space approaches. 

\subsubsection{Pattern/Rule Based Approaches}

\citet{hearst1992} first introduced ontology learning by exploiting lexico-syntactic patterns that explicitly link a hypernym to its hyponym, e.g.  ``\emph{X and other Ys}" and ``\emph{Ys such as X}". These patterns could be manually constructed \citep{berland1999finding,kozareva2008} or automatically bootstrapped \citep{girju2003automatic}. These methods rely on surface-level patterns and incorrect items are frequently extracted because of parsing errors, polysemy, idiomatic expressions, etc.

In the recent taxonomy induction shared tasks at SemEval \citep{task13semeval2016,bordea-EtAl:2015:SemEval}, rule based systems using substring heuristics have gained popularity and shown to attain high precision across multiple domains \citep[e.g.][]{lefever:2015:SemEval,panchenko2016taxi,usaarsemeval2016}. However, these systems still suffer from low recall but the precision-recall trade off favors rule-based systems due to the high false positive rates produced by non-rule based systems \citep[e.g.][]{nuigsemeval2016} that have achieved comparative recall rates as the rule-based ones.

\subsubsection{Clustering Approaches}

Clustering based approaches are mostly used to discover hypernym (is-a) and synonym (is-like) relations. For instance, to induce synonyms, \citet{lin1998automatic} clustered words based on the amount of information needed to state the commonality between two words.\footnote{Commonly known as Lin information content measure.}

\cite{Caraballo1999} was first to combine clustering and pattern-based methods by hierarchically clustering words and assigning the hypernyms by identifying the ``\emph{A is a (kind of) B}" and ``\emph{X, Y, and other Zs}" patterns where B is considered as a hypernym of A and Z is the hypernym of X and Y. 

Contrary to most bottom-up clustering approaches for taxonomy induction \citep{Caraballo2001,lin1998automatic}, \citet{pantelravi2004} introduced a top-down approach, assigning the hypernyms to clusters using co-occurrence statistics and then pruning the cluster by recalculating the pairwise similarity between every hyponym pair within the cluster.

Besides inducing generic hypernym-hyponym taxonomies, similar clustering approaches were also applied in inducing domain-specific knowledge bases that focused on inferring relations between named entities \citep[e.g.][]{hasegawa2004discovering}.


\subsubsection{Graph-based Approaches}

In graph theory \citep{biggs1976graph}, similar ideas are conceived with a different jargon. In graph notation, \emph{\textbf{nodes}}/\emph{\textbf{vertices}} form the atom units of the graph and nodes are connected by directed \emph{\textbf{edges}}. A \emph{graph}, unlike an ontology, regards the hierarchical structure of a taxonomy as a by-product of the individual pairs of \emph{nodes} connected by directed \emph{edges}. In this regard, a single \emph{root} node is not guaranteed nor a tree-like structure. 

Disregarding the overall hierarchical structure, the crux of graph induction focuses on the different techniques of edge weighting between individual node pairs and graph pruning or edge collapsing \citep{kozareva2010semi,navigliVF11,fountain2012taxonomy,Tuan2014}. 

\subsubsection{Vector Space Approaches}

Semantic knowledge can be thought of as a two-dimensional vector space where each word is represented as a point and semantic association is indicated by word proximity. The vector space representation for each word is constructed from the distribution of words across context, such that words with similar meaning are found close to each other in the space \citep{Mitchell:Lapata:2010,xling2013}.

Although vector space models have been used widely in other NLP tasks, ontology/taxonomy induction using vector space models has not been popular. It is only since the recent advancement in neural nets and word embeddings that vector space models are gaining ground for ontology induction and relation extraction \citep{saxe2013,khashabi2013}.

\subsubsection{Projecting a Hyponym to its Hypernym with a Transition Matrix}
\citet{fu2014semhierarchies} discovered that hypernym-hyponyms pairs have similar semantic properties as the linguistic regularities discussed in \citet{mikolov2013}. For instance: 
\\ \\
\centerline{\emph{v}\texttt{(shrimp)}-\emph{v}\texttt{(prawn)} $\approx$ \emph{v}\texttt{(fish)}-\emph{v}\texttt{(goldfish)}}
\\ \\
\noindent Intuitively, the assumption is that all words can be projected to their hypernyms based on a transition matrix. That is, given a word \emph{x} and its hypernym \emph{y}, a transition matrix $\Phi$ exists such that 
\vspace{2mm}
\\ \\
\centerline{\emph{y} = $\Phi$\emph{x}, e.g. \emph{v}\texttt{(goldfish)} = $\Phi$$\times$\emph{v}\texttt{(fish)}}
\\ \\
\noindent Fu et al. proposed two projection approaches to identify hypernym-hyponym pairs, (i) uniform linear projection where $\Phi$ is the same for all words and $\Phi$ is learnt by minimizing the mean squared error of $\|$$\Phi$\emph{x}-\emph{y}$\|$ across all word-pairs (i.e. a domain independent $\Phi$) and (ii) piecewise linear projection that learns a separate projection for different word clusters (i.e. a domain dependent $\Phi$, where a taxonomy's domain is bounded by its terms' cluster(s)). In both projections, hypernym-hyponym pairs are required to train the transition matrix $\Phi$.

\section{Ontology and Translation}

Another aspect of semantic knowledge within translation is the usage of ontology in human and machine translation. Beyond the flat structure of a list of domain-specific words and phrases in the terminology, an ontology provides a hierarchical structure between the words. The resulting `tree of domain-specific knowledge' is helpful for human translation when the translators are not familiar with the domain. 

The goal of ontology development is the sharing of common knowledge of the information structure across different departments working in the same organization \citep{MUSEN1992435,gruber1993translation}. Although humanly crafted broad-based ontologies (i.e. upper ontologies) that attempt to catch the \textit{world knowledge} exist  \citep{lenat1995cyc,PeaseEtAl2002sumo,navigli2012babelnet}, it can be more viable to automatically create a domain-specific ontology given a domain-specific corpus in the target language. In this thesis we introduce (i) \textbf{\textit{a novel state-of-art technique to generate hypernyms between terms to create an ontology using neural net embeddings}} and (ii) \textbf{\textit{explore the endocentricity of hyper-hyponyms relations based on their surface string representation}}. This will be discussed in Chapter 6.

We hope that the techniques and findings we have on ontology induction can be further developed beyond the scope of the thesis to support automatic ontology creation to help translators in their knowledge understanding part of their translation workflow.

Beyond aiding human translators, ontology can be provide additional knowledge to the statistical machine translation training process. A simple way to incorporate ontological knowledge is to develop word clusters instead of a full hierarchical knowledge graph. Within the word alignment phase of the SMT training steps, there is a module that performs word clustering in order to reduce sparsity of the relative distortion computation\footnote{The discussion on word alignment algorithm is out of scope in this thesis but the Moses developers have an informational deck of slides describing it at \url{http://www.statmt.org/book/slides/04-word-based-models.pdf} \citep{koehn2009statistical}} \citep{och2003systematic}. In this thesis we present joint work on the \textit{\textbf{evaluation of a new predictive exchange clustering algorithm that can be used to replace  existing clustering software}}, {\tt mkcls} within the (M)GIZA++ word alignment tool that is used in the phrase-based machine translation. This will be discussed in Chapter 6.


\section{Summary}

Integrating additional lexical information into statistical machine translation is nothing new but the marginal gains in BLEU scores across the literature is somewhat disturbing given the industrial importance of translating terminologies correctly. This thesis seeks to take a closer look at how to use additional lexical resources in the SMT training process. 

Ontologies are useful knowledge resources that translators use to familiarize themselves with domain specific knowledge. The manual creation of ontology is time and labor consuming, this thesis proposes a novel approach to ontology induction using neural nets and explores the endocentricity of hypernyms within the ontology to better understand how we can effectively automatically induce high precision ontologies.

The lack of `semantic knowledge' in the statistical machine translation paradigm suggests that we should look into ways to inject meaning into the probabilistic system. In this thesis, we will present joint work on scaling sub-ontological knowledge (i.e. word clusters) and investigate the improvements made by this to statistical machine translation.






