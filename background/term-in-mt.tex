\newpage
\subsection{The Importance of Terminology in Machine Translation}

In today`s globalized world, the ability to localize information into a foreign market is crucial to business expansion and machine translation (MT) is the an important means to help translate the sheer amount of information that global businesses need to process daily.

Businesses benefit from translating documents automatically by accelerating corporate communication and an MT system sensitive to domain-specific terminology is crucial in ensuring uniform and clear corporate language \citep{porsiel2011}. Yet ``\emph{terminology is the biggest factor in poor translation quality}" \citep{warburton2005} and ``\emph{businesses often fail to see terminology management as a way to cut costs}" \citep{clientsidenews2006}. \cite{lionbridge2010} reported, ``\emph{approximately 15 percent of all globalization project costs arise from rework, and the primary cause of rework is inconsistent terminology".}

The use of MT is worthwhile if the following prerequisites are given: there must be a specific corporate terminology in the largest possible scope and of the best possible quality in both the source and target languages \citep{porsiel2008}.  As highlighted, the two main points in ensuring that terminology is useful for improving MT requires (i) the largest possible scope and of best possible quality, i.e. \emph{recall and precision} and (ii) in both source and target languages, i.e. \emph{bilingual}.  

In 2008, the Swedish car manufacturer Volvo was found partly to blame for a car accident which killed two school children. The expert engineer appointed by the court criticized the poor translation of a Third-Party Intermediary in the car manual on power-assisted brakes, the expert engineer stated, `\emph{... there is some cause for highlighting the fact ... that the technical product information EWP S 2631 (D 710) issued by the manufacturer and translated by the import was imprecise and poorly written}`. Volvo was fined 200,000 Euros for involuntary manslaughter and bodily injury (c.f. \citealp{hoffmeister2014}). This called for a global concerns when handling terminology translations in technical manuals, especially in current translation workflows that incorporate machine translation with human post-editing. 

