\documentclass[11pt]{article}

\usepackage{hyperref}
\usepackage{graphicx}
\usepackage{verbatim}
\usepackage[round]{natbib}
\usepackage{color}
\usepackage{setspace}
\usepackage{enumerate}
\usepackage{enumitem}
\usepackage{multicol}
\usepackage[titletoc]{appendix}
\usepackage{lscape}
\usepackage{enumitem}
\usepackage{amsmath}
\usepackage{csquotes}


%%%%%%%%%%%%%%%%%%%%%%%%%%%%%%%%%%%%%
% Packages Language
%%%%%%%%%%%%%%%%%%%%%%%%%%%%%%%%%%%%%
\usepackage{CJKutf8}
\usepackage[T1]{fontenc}

\newenvironment{korean}{%
  \CJKfamily{mj}}{}

\usepackage{times}
\usepackage{latexsym}

%%%%%%%%%%%%%%%%%%%%%%%%%%%%%%%%%%%%%
% Packages for presenting XML code
%%%%%%%%%%%%%%%%%%%%%%%%%%%%%%%%%%%%%

\usepackage{listings}
\usepackage{fancyvrb}
\usepackage{color}
\definecolor{gray}{rgb}{0.4,0.4,0.4}
\definecolor{darkblue}{rgb}{0.0,0.0,0.6}
\definecolor{cyan}{rgb}{0.0,0.6,0.6}
\definecolor{orange}{rgb}{1,0.5,0}

\lstset{
  basicstyle=\ttfamily,
  columns=fullflexible,
  showstringspaces=false,
  commentstyle=\color{gray}\upshape
}


\lstdefinelanguage{XML}
{
  morestring=[b]",
  morestring=[s]{>}{<},
  morecomment=[s]{<?}{?>},
  stringstyle=\color{black},
  identifierstyle=\color{darkblue},
  keywordstyle=\color{cyan},
  morekeywords={xmlns,version,type}% list your attributes here
}

%%%%%%%%%%%%%%%%%%%%%%%%%%%%%%%%%%%%%
% Commands for margin comments
%%%%%%%%%%%%%%%%%%%%%%%%%%%%%%%%%%%%%

\newcommand{\fb}[1]{\marginpar{\footnotesize \textbf{FB:} #1}}
\newcommand{\jvg}[1]{\marginpar{\footnotesize \textbf{JvG:} #1}}
\newcommand{\lt}[1]{\textbf{LL:} \textcolor{red}{#1}}

%%%%%%%%%%%%%%%%%%%%%%%%%%%%%%%%%%%%%
% Custom commands
%%%%%%%%%%%%%%%%%%%%%%%%%%%%%%%%%%%%%

% Customize forced tab.
\newcommand{\tab}[1]{\hspace{.2\textwidth}\rlap{#1}}

% Customized argmin operator
\DeclareMathOperator*{\argmin}{arg\,min}
% Customized argmin operator
\DeclareMathOperator*{\argmax}{arg\,max}


\title{\textbf{Panda Hyper Decoder}}
\author{Liling Tan}


\date{}
\begin{document}

\maketitle

%%%%%%%%%%%%%%%%%%%%%%%%%%%%%%%%%%%%%%%%%%%%%%%%%%%%%%%%%%%%%%%%%%%%%%%%%%%
% CJK Examples.
%%%%%%%%%%%%%%%%%%%%%%%%%%%%%%%%%%%%%%%%%%%%%%%%%%%%%%%%%%%%%%%%%%%%%%%%%%%
\newpage
\section{CJK Examples}

\subsection{Chinese}
\textbf{ZH Input}: \begin{CJK*}{UTF8}{gbsn}应有尽有 的 丰富 选择 定 将 为 您 的 旅程 增添 无数 的 赏 心 乐事\end{CJK*}

\subsection{Japanese}
\textbf{JP Input}: \begin{CJK*}{UTF8}{min}種の量が増加する程,生成物の粒子寸法は減少した。\end{CJK*}
\\ \\
(Hiragana \begin{CJK*}{UTF8}{min}ひらがな\end{CJK*} and Katakana \begin{CJK*}{UTF8}{min}カタカナ\end{CJK*}) and logographic (Kanji \begin{CJK*}{UTF8}{min}漢字\end{CJK*})

\subsection{Korean}
\textbf{KR Input}: \begin{CJK*}{UTF8}{}\begin{korean}싱가포르에서 가장 유명한 하이난식 치킨 라이스 음식점 중 하나인 Tian Tian Hainanese Chicken Rice 는 맥스웰 푸드 센터 (Maxwell Food Centre) 에 위치해 있으며, 음식점 앞에는 손님들이 매일 길게 줄지어 서 있습니다. \end{korean} \end{CJK*}

%%%%%%%%%%%%%%%%%%%%%%%%%%%%%%%%%%%%%%%%%%%%%%%%%%%%%%%%%%%%%%%%%%%%%%%%%%
% Comment Examples
%%%%%%%%%%%%%%%%%%%%%%%%%%%%%%%%%%%%%%%%%%%%%%%%%%%%%%%%%%%%%%%%%%%%%%%%%%
\newpage
\section{Comment Examples}
You will see this
% but not this
\iffalse
You won't see this example
\fi
\jvg{this will appear at the side}
\fb{so will this}
\\ \\
\lt{this appears in line in red}

%%%%%%%%%%%%%%%%%%%%%%%%%%%%%%%%%%%%%%%%%%%%%%%%%%%%%%%%%%%%%%%%%%%%%%%%%%%
% Equation Examples.
%%%%%%%%%%%%%%%%%%%%%%%%%%%%%%%%%%%%%%%%%%%%%%%%%%%%%%%%%%%%%%%%%%%%%%%%%%%
\newpage
\section{Equation Examples}
\subsection{Models}
We used a lightweight Naive Bayes classification model that was previously described in language identification studies. Naive Bayes is a popular classification model due to its robustness and speed. The language of test document \emph{D} is predicted by maximizing the sum of the logarithmic probability of a feature (i.e. word/character ngrams frequency) \emph{w} given a language \emph{l}:

\begin{equation}
\hat{l}(D) = \underset { { l }_{ i }\epsilon L }{ { argmax } } \sum _{ j=1 }^{ |V| }{ \log { P({ w }_{ j }|{ l }_{ i }) }  }
\end{equation}
\vspace{0.1 mm}

\noindent where L is the set of languages/varieties in each language group, \emph{N} is the frequency of the \emph{j}th word/character ngram in \emph{D} and \emph{V} is the set of all word/character ngrams in the training data. We use the \texttt{sklearn} implementation of multinomial Naive Bayes in our experiments\footnote{www.scikit-learn.org},  which calculates:

\begin{equation}
P({ w }|{ l }_{ i })=\frac { \sum _{ k=1 }^{ |\delta | }{ { N }_{ k,w } } + \alpha  }{ |V| +\sum _{ j=1 }^{ |V| }{ \sum _{ k=1 }^{ |\delta | }{ { N }_{ k,{ w }_{ j } } }  }  } 
\end{equation}
\vspace{0.1 mm}

\noindent where $\delta$ is the set of features from the test document D and $\alpha$ is the smoothing factor; setting $\alpha$=1 results in Laplace smoothing and $\alpha$\textless1 for Lidstone smoothing. We used Laplace smoothing for our experiments. 


These basic statistical properties of word co-occurrence are combined in various ways to form more complex co-occurrence measures. The common co-occurrence measures are defined as follows:

\begin{eqnarray} 
Dice(i,j) & = & \frac { 2*a }{ ({ f }_{ i }+{ f }_{ j }) }  \\[1em]
PMI(i,j) & = & \log { a } -(\log { { f }_{ i } } +\log { { f }_{ j }) }  \\ [1em]
MI(i,j) & = & \log { a } -\log { (a+b) } -\log { (a+c) }  \\ [1em]
LLR(i,j) & = & a*\log { a } +b*\log { b } +c*\log { c } +d*\log { d }  \nonumber\\  
&  & -(a+b)*\log { (a+b) } -(a+c)*\log { (a+c) }  \nonumber\\  
&  & -(b+d)*\log { (b+d) } -(b+d)*\log { (c+d) }  \nonumber\\  
&  & +(a+b+c+d)*\log { (a+b+c+d) }  \\ [1em]
{ \phi  }^{ 2 } & = & \frac { (ad–bc)^{ 2 } }{ (a+b)(a+c)(b+c)(b+d) }  
\end{eqnarray} 

Given the definition of \emph{T$_N$}, we calculate the \emph{C-Value} of a term $t$ as follows:

\begin{equation}
C-value(t)=
	\left \{
		\begin{matrix} 
		\log { |t| } *{ f }_{ t }  & if \ t \ is \ nested\\
		\log { |t| } *{ (f }_{ t }-{ 1 }/{ { f }_{ { t }_{ N } } }*
			\sum _{ i\in { T }_{ N } }^{  }{ { f }_{ i } }  & otherwise
		\end{matrix} 
	\right .
\end{equation}

%%%%%%%%%%%%%%%%%%%%%%%%%%%%%%%%%%%%%%%%%%%%%%%%%%%%%%%%%%%%%%%%%%%%%%%%%%%
% XML Examples.
%%%%%%%%%%%%%%%%%%%%%%%%%%%%%%%%%%%%%%%%%%%%%%%%%%%%%%%%%%%%%%%%%%%%%%%%%%%
\newpage
\section{XML Examples}

Here's an XML example in latex:

\begin{figure}[!htb]
\small
\begin{lstlisting}[language=XML]
<igt>
  <example>
    <line>21 a. o lesu mai</line>
    <line>2sg return here</line>
    <line>`You return here.'</line>
  </example>
</igt>
\end{lstlisting} 
\caption{Fijian IGT in ODIN's XML format} \label{fig:odin_fijian_xml}
\end{figure}

%%%%%%%%%%%%%%%%%%%%%%%%%%%%%%%%%%%%%%%%%%%%%%%%%%%%%%%%%%%%%%%%%%%%%%%%%%%
% Block Quotes Examples.
%%%%%%%%%%%%%%%%%%%%%%%%%%%%%%%%%%%%%%%%%%%%%%%%%%%%%%%%%%%%%%%%%%%%%%%%%%%
\newpage
\section{Block Quote Examples}

Something something...

\blockquote{The past is not a peaceful landscape lying there behind me, 
a country in which I can stroll wherever I please, 
and will gradually show me all its secret hills and dates. 
As I was moving forward, so it was crumbling. 
Some more text to make it longer than three lines. 
Maybe it is long enough now.}


%%%%%%%%%%%%%%%%%%%%%%%%%%%%%%%%%%%%%%%%%%%%%%%%%%%%%%%%%%%%%%%%%%%%%%%%%%%
% Acknowledge Examples.
%%%%%%%%%%%%%%%%%%%%%%%%%%%%%%%%%%%%%%%%%%%%%%%%%%%%%%%%%%%%%%%%%%%%%%%%%%%
\newpage
\section*{Acknowledgements}

The research leading to these results has received funding from the People Programme (Marie Curie Actions) of the European Union's Seventh Framework Programme FP7/2007-2013/ under REA grant agreement n$\,^{\circ}$ 317471.

\end{document}
