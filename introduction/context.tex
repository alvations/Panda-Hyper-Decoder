
\section{Context: The EXPERT Project}

The research reported in this PhD thesis was carried out under the EXPERT (EXPloiting Empirical appRoaches to Translation) project. EXPERT is a European Union's (EU)  Seventh Framework Programme FP7/2007-2013/ Marie-Curie Initial Training Network (ITN) under REA grant agreement no.~317471. The project is concerned with the exploitation of empirical approaches to provide innovation to the field of Translation Memory (TM) and Machine Translation (MT) technologies.

Under the project, the consortium of academic and industrial partners are assigned various translation related research topics; these includes:

\vspace{5mm}
\begin{itemize}[nosep]
\item[1.] Investigation of translators' requirements from translation technologies
\item[2.] Investigation of an ideal translation workflow for hybrid translation approaches
\item[3.] Collection and preparation of multilingual data for multiple corpus-based approaches to translations
\item[4.] Usage of language technology to improve matching and retrieval in translation memories (TMs)
\item[5.] Usage of terminologies and ontologies to improve translation
\item[6.] Learning from human feedback on the quality of the translations
\item[7.] Estimating the confidence and quality of corpus-based approaches to translation
\item[8.] Investigation of how each individual corpus-based translation approach (TM, EBMT and SMT) can benefit from each other
\item[9.] Integration of NLP technologies into tools for Computer Aided Translation
\item[10.] Exploiting hierarchical alignments for linguistically-informed SMT models to meet the hybrid approaches that aim at compositional translation
\item[11.] Exploiting hierarchical alignments for a semantically-enriched SMT system that offers an extension to existing TMs to allow incremental, recursive partial match of the input
\item[12.] Innovations to Machine Translation Evaluation 
\end{itemize}

This thesis reports the findings of the research carried out under (5) \textit{usage of terminologies and ontologies to improve translation} and presents novel ideas to terminology and ontology construction in relation to machine translation technologies \citep{pervasive2015,tan-gupta-vangenabith:2015:SemEval}. 

